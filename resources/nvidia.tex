\textbf{Senior System Software Engineer}\\
NVIDIA Corporation, Santa Clara, CA and Austin, TX \hfill 2011-2016

\begin{itemize} \itemsep -2pt

\item Firmware and device driver development for resource management, 2D graphics, 3D graphics, and accelerated video decode for NVIDIA desktop/mobile GPUs on Linux/UNIX hosts.

\item Technical lead for VDPAU, the accelerated hardware video decoder stack on Linux.

\item Technical lead for H.265/HEVC video decoder format implementation on Linux. Added hardware accelerated H.265/HEVC video playback support to the NVIDIA Linux driver stack. Wrote \url{https://github.com/NVIDIA/vdpau-hevc-example}, a simple stream parser to play H.265/HEVC video released under an open source license.

\item Worked with firmware, kernel, hardware, and ASIC engineers around the clock and around the world on FPGA Quickturn emulators to root cause and fix a critical, stop ship issue related to video decoding on Maxwell GPUs.


% Hard code a page break here.
\end{itemize}

\newpage
% There should be a page break here.

{ \sffamily Jos\'e Hiram Soltren} \\

\section{Experience, continued}


\begin{itemize} \itemsep -2pt
\item Refactored the NVIDIA Linux video decode driver stack to add hardware support for Maxwell and Pascal GPUs. Worked with firmware, hardware, and ASIC engineers to sort out hardware differences from previous generations.

\item Rewrote video shaders used in the video decode pipeline from Cg into GLSL, allowing the video shaders to move to a leaner implementation using a new shader compiler.

% \item Served as company-wide expert and point person for H.264 and H.265 format questions. Helped to set direction for video decoding hardware in future NVIDIA hardware products.

% \item Worked on the critical path for new hardware bringup. Designed and executed bringup tests for 2D and 3D graphics and video decoding. Worked to right-size and automate bringup testing, with a direct result of shortened test times and faster time to market. Ported these improvements to pre-silicon early validation tests to shorten delivery times for future hardware generations.

%\item Root caused and fixed a critical bug that would sometimes cause 3D driver initialization to fail. This bug, when fixed, unblocked content creation for a major motion picture studio.

%\item Fixed a critical video decoding issue that would result in system hangs when decoding corrupted video streams. This bug, when fixed,  enabled hospital patients to enjoy a first-class video playback experience on nearly end-of-lifed hardware.



%\item Discovered, root caused, and fixed a critical race condition bug that prevented the NVIDIA Linux driver from performing reliably with more than four GPUs in a system, by introducing a fine grained lock in a crucial code path. Fixing this bug made it possible to place multiple GPUs into one system, an important workflow for machine learning. In so doing, implemented a novel debugger that worked closely with the kernel to register hardware breakpoints after certain conditions were met.

%\item Worked closely with kernel engineers on hwsnoop, an in-system PCIe debugger. It could, via PCI bus mastery, investigate the state of a halted CPU or GPU, so long as the system's memory controller remained active. Used hwsnoop to gather information on system crash type bugs that could not be investigated any other way.

%\item Performance and QA work for hardware and driver support on Google's ChromeOS.

%\item Discovered a number of potential stop-ship issues that slipped by our QA teams. Fixed them even if they were outside my domain.

%\item Authored and maintained tools for bug triage and bringup testing.

%\item Worked with customer facing QA teams to improve practices, leading to reduced engineering time spent on issues.

\item Fixed multiple bugs in the NVIDIA Linux driver graphics stack related to performance, system reliability, video quality, and memory economy.

\item Served as technical mentor to several engineers over the years. Mentees have gone on to make substantial contributions to OpenGL, Vulkan, X.org, and the Linux kernel.

\item \textit{Technologies used: C, C++, OpenGL, Cg, GLSL, assembly (numerous variants), FPGA hardware emulation, Linux kernel, gdb, Perforce, git}

\end{itemize}
